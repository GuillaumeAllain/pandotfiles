\pagestyle{empty}

\usepackage{fancyhdr}
\usepackage[top=2.5cm, bottom=4cm, left=2.5cm, right=2.5cm,marginparwidth=3.6cm, marginparsep=3mm, headsep=1cm, footskip=40pt]{geometry}

% Nombres et bibliographie
\ifxetex
  \setmainfont{Palatino}
\fi
\pagenumbering{arabic}

% Titrage
\usepackage{blindtext}
\usepackage{sectsty}
\usepackage{xparse}
\usepackage{lastpage}

% Bibliographie et acronymes
\usepackage{multibib}
\newcites{internal}{Documents internes}
% \bibliographystyle{unsrtnat}
\bibliographystyleinternal{unsrtnat}

%%%%%%%%%%%%%%%%% STYLE %%%%%%%%%%%%%%%%%%%%%%%%%%%%%%%%%%%

%lengtes voor titelpagina
\newlength{\logowidth}
\setlength{\logowidth}{5cm}
\setlength\parindent{0pt}
\setlength{\parskip}{0.5\baselineskip}
% \usepackage{tocloft}
% \renewcommand\cftsecafterpnum{\vskip15pt}

\pagestyle{fancy}

\addtolength{\headheight}{30pt}
\renewcommand{\headrulewidth}{0.5pt}
\renewcommand{\footrulewidth}{0.5pt}
\fancyhead{}
\fancyhead[L]{\footnotesize\includegraphics[height=1cm]{pandot_img/logo_GA.png}}
$if(title)$
\fancyhead[R]{\rule[-2.5ex]{0pt}{2.5ex}\footnotesize $title$}
$endif$
\fancyfoot{}
$if(author)$
  \fancyfoot[L]{\rule[1.5ex]{0pt}{1.5ex}\footnotesize $author$}
$endif$
\fancyfoot[R]{\rule[1.5ex]{0pt}{1.5ex}\footnotesize \thepage/\pageref*{LastPage}}
$if(website)$
  \fancyfoot[C]{\rule[-2.5ex]{0pt}{2.5ex}\footnotesize $website$}
$endif$


%commando om titelpagina te maken
\renewcommand{\maketitle}{
  \thispagestyle{fancy}
  \begin{center}
    \ifthenelse{\equal{\titleprint}{}}{
      }{
      \huge\textbf{\titleprint}\\\vspace{0.5cm}
    }
    \ifthenelse{\equal{\subtitleprint}{}}{
      }{
      \large{\subtitleprint}\\\vspace{0.5cm}
    }
    \ifthenelse{\equal{\authorprint}{}}{
      }{
      \Large{\authorprint}\\\vspace{0.5cm}
    }
    \ifthenelse{\equal{\dateprint}{}}{
      }{
      \normalsize{\dateprint}
    }
  \end{center}
  % \printglossary[type=\acronymtype,title=Acronymes]
  \ifthenelse{\equal{\acronyms}{}}{}{
  \printglossary[type=\acronymtype,title=Acronymes]
  }
  \ifthenelse{\equal{\bibliographyinternefile}{}}{}{
    % \newcites{interne}{Documents internes}
    % \bibliographystyleinterne{unsrtnat}
    \bibliographyinternal{\bibliographyinternefile}
  % \printbibliography[title=interne, keyword=secondary]
  }
}


\usepackage{hyperref}
\hypersetup{colorlinks, allcolors=[rgb]{0, 0, 0.3}}
