\providecommand{\tightlist}{%
  \setlength{\itemsep}{0pt}\setlength{\parskip}{0pt}}
\ifthenelse{\equal{$spielineno$}{}}{}{
  \usepackage{lineno}
}
\usepackage{amsmath,amsfonts,amssymb}
\usepackage{longtable}
\usepackage{booktabs}

\ifthenelse{\equal{$spielineno$}{}}{}{
%% Patch 'normal' math environments:
\usepackage{etoolbox} %% <- for \pretocmd, \apptocmd and \patchcmd
\newcommand*\linenomathpatch[1]{%
  \cspreto{#1}{\linenomath}%
  \cspreto{#1*}{\linenomath}%
  \csappto{end#1}{\endlinenomath}%
  \csappto{end#1*}{\endlinenomath}%
}
%% Patch AMS math environments:
\newcommand*\linenomathpatchAMS[1]{%
  \cspreto{#1}{\linenomathAMS}%
  \cspreto{#1*}{\linenomathAMS}%
  \csappto{end#1}{\endlinenomath}%
  \csappto{end#1*}{\endlinenomath}%
}

%% Definition of \linenomathAMS depends on whether the mathlines option is provided
\expandafter\ifx\linenomath\linenomathWithnumbers
  \let\linenomathAMS\linenomathWithnumbers
  %% The following line gets rid of an extra line numbers at the bottom:
  \patchcmd\linenomathAMS{\advance\postdisplaypenalty\linenopenalty}{}{}{}
\else
  \let\linenomathAMS\linenomathNonumbers
\fi

\linenomathpatch{equation}
\linenomathpatchAMS{gather}
\linenomathpatchAMS{multline}
\linenomathpatchAMS{align}
\linenomathpatchAMS{alignat}
\linenomathpatchAMS{flalign}
}
\usepackage{graphicx}
\makeatletter
\newsavebox\pandoc@box
\newcommand*\pandocbounded[1]{% scales image to fit in text height/width
  \sbox\pandoc@box{#1}%
  \Gscale@div\@tempa{\textheight}{\dimexpr\ht\pandoc@box+\dp\pandoc@box\relax}%
  \Gscale@div\@tempb{\linewidth}{\wd\pandoc@box}%
  \ifdim\@tempb\p@<\@tempa\p@\let\@tempa\@tempb\fi% select the smaller of both
  \ifdim\@tempa\p@<\p@\scalebox{\@tempa}{\usebox\pandoc@box}%
  \else\usebox{\pandoc@box}%
  \fi%
}
\usepackage{siunitx}
\usepackage[colorlinks=true, allcolors=blue]{hyperref}
\renewcommand{\baselinestretch}{1.0} % Change to 1.65 for double spacing
\makeatletter
  $if(float_placement)$\def\fps@figure{ $float_placement$}$endif$
  $if(float_placement)$\def\fps@table{ $float_placement$}$endif$
\makeatother
 
\title{$title$}
$for(author)$
\author[$if(author.affil)$
  $author.affil$
  $endif$]{
$if(author.name)$
$author.name$
$else$
$author$
$endif$}
$endfor$

$for(affil)$
\affil[$affil.id$]{$affil.content$}
$endfor$
\ifthenelse{\equal{$authorinfo$}{}}{
   \authorinfo{authorinfo ex:\\Further author information: (Send correspondence to A.A.A.)\\A.A.A.: E-mail: aaa@tbk2.edu, Telephone: 1 505 123 1234\\  B.B.A.: E-mail: bba@cmp.com, Telephone: +33 (0)1 98 76 54 32} %% email address is required
}{
   \authorinfo{$authorinfo$} %% email address is required
}

% \affil[a]{Affiliation1, Address, City, Country}
% \renewcommand{\maketitle}{



    % \author{\authorprint}

    % \ifthenelse{\equal{$spieaddress$}{}}{
    %     \email{address is needed} %% email address is required
    % }{
    %     \email{\spieaddress} %% email address is required
    % }


% }

\bibliographystyle{spiebib} % makes bibtex use spiebib.bst

% \title{SPIE Proceedings: Style template and guidelines for authors}

% \author[a]{Anna A. Author}
% \author[b]{Barry B. Author}
% \affil[a]{Affiliation1, Address, City, Country}
% \affil[b]{Affiliation2, Address, City, Country}

% Option to view page numbers
\pagestyle{empty} % change to \pagestyle{plain} for page numbers   
% \setcounter{page}{301} % Set start page numbering at e.g. 301
\ifthenelse{\equal{$spielineno$}{}}{}{
  \linenumbers
}
